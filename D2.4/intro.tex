\section{Introduction}

The purpose of this document is to describe, for the OpenETCS project, the means and methods used to perform the formal description.

However the benchmark activities are not yet achieved in WP7, such the definition of the methods can not yet be done.

For the intermediate version of the document, we propose a description of the benefits of formal methods in the design, development, verification and validation of critical systems.

For the final version, the document proposes the method to  follow during the OpenETCS project according to  the OpenETCS process  and requirements defined in D2.3  and D2.6.

\section{Reference documents}
\label{sec:ref}

\begin{itemize}
\item CENELEC EN 50126-1 --- 01/2000 --- \emph{Railways applications –- The specification and 
demonstration of Reliability, Availability, Maintenability and Safety (RAMS) –- Part 1: 
Basic requirements and generic process}
\item CENELEC EN 50128 --- 10/2011 --- \emph{Railway applications -- Communication, signalling and 
processing systems -- Software for railway control and protection systems}
\item CENELEC EN 50129 --- 05/2003 --- \emph{Railway applications –- Communication, signalling and 
processing systems –- Safety related electronic systems for signalling}
\item FPP --- \emph{Project Outline Full Project Proposal Annex OpenETCS} -- v2.2
\item SUBSET-026 3.3.0 --- \emph{System Requirement Specification}
\item SUBSET-076-x 2.3.y --- Test related ERTMS documentation
\item SUBSET-088 2.3.0 --- \emph{ETCS Application Levels 1 \& 2 - Safety Analysis}
\item SUBSET-091 2.5.0 --- \emph{Safety Requirements for the Technical Interoperability
of ETCS in Levels 1 \& 2}
\item CCS TSI --- \emph{ CCS TSI for HS and CR transeuropean rail has been adopted by a Commission Decision 2012/88/EU on the 25th January 2012}
\item D1.3 -- Project Quality Assurance Plan
\item D2.1 -- Report on existing methodologies 
\item D2.2 -- Report on CENELEC standards
\item D2.3 -- Definition of the overall process for the formal description of ETCS and the rail system it works in 
\item D2.6 -- Requirements for OpenETCS
\end{itemize}


%%%%%%%%%%%%%%%%%%%%%%%%%%%%%%%%%%%%%%%%%%%%%%%%%%%%%%%%%%%%%%%

\section{Glossary}
\label{sec:glossary}

\begin{description}
\item[API] Application Programming Interface
\item[FME(C)A] Failure Mode Effect (and Criticity) Analysis
\item[FIS] Functional Interface Specification
\item[HW] Hardware
\item[I/O] Input/Output
\item[OBU] On-Board Unit
\item[PHA] Preliminary Hazard Analysis
\item[QA] Quality Analysis
\item[RBC] Radio Block Center
\item[RTM] RunTime Model
\item[SIL] Safety Integrity Level
\item[SRS] System Requirement Specification
\item[SSHA] Sub-System Hazard Analysis
\item[SSRS] Sub-System Requirement Specification
\item[SW] Software
\item[THR] Tolerable Hazard Rate
\item[V\&V] Verification \& Validation
\end{description}
