
\section{Guidelines on the approaches used for OpenETCS}


\begin{comment}
This section will be written for the final version of the document, after the approach and tools tio  use during the project will be selected.
\end{comment}

According to  the WP7 decision meeting, the 4th of July, in Paris, SysML, supported by the Papyrus tool,  has been chosen to  cover the highest level of modelling.

The choice of the approaches for the lower levels of modelling is not yet fixed.

This section gives a proposal on how to use the selected approaches to produce from the input documents (ERA documentation and complements)  to a SIL4 code.

\subsection{Sum up of chosen approaches}
\begin{comment}
list of chosen approaches, and for which activities they are used
\end{comment}


\subsection{Artifacts and common items}

\begin{comment}
list of artifacts used and provided by each approach; common items (data model); types
\end{comment}


\subsection{Requirement naming convention}


\begin{comment}
TODO: Marc Behrens
\end{comment}


\subsection{Name convention}


\begin{comment}
How to name object ? base : subset 26 §7.3.2:
" 7.3.2.11 All Variables have one of the following prefixes:
\begin{itemize}
\item A\_ Acceleration
\item D\_ distance
\item G\_ Gradient
\item L\_ length
\item M\_ Miscellaneous
\item N\_ Number
\item NC\_ class number
\item NID\_ identity number
\item Q\_ Qualifier
\item T\_ time/date
\item V\_ Speed
\item X\_ Text
\end{itemize}


\end{comment}


\begin{comment}
Case sensitive language, keywords of target language (SysML, B, C, Scade 5?),...), UPPER case for names issued form official document (SRS, API,...), lower case for others. length of the name

QA plan to check.
\end{comment}

