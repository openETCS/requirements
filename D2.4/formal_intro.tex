

\section{Short introduction on formal approaches to  design and validate critical systems}

\subsection{What is a formal approach?}

A \emph{formal} approach is a way to describe system or software
that builds upon (i) rigorous syntax and (ii) rigorous semantics.

The \emph{syntax} defines how the system or software is described. It
is usually made through a grammar and a set of additional
constraints. It can be textual or graphical.

The \emph{semantics} roughly describes how the system or software
evolves along time. It is defined using a mathematical model, i.e. use
of mathematical objects attached to each element of the syntax and
mathematical rules that define how those objects evolve. The
mathematical models used can be very different from a formal approach
to another one. For example B~Method uses the Generalized
Substitutions, SCADE relies on the Synchronous language Lustre, etc.

A \emph{semi-formal} approach is one where the syntax is precisely
defined but the semantics is not precisely defined, usually through
some English text. Typical semi-formal approaches are Matlab language
or SysML/UML formalisms.

A semi-formal approach can become formal if its semantics is
rigorously defined through a mathematical model.

\subsection{When formal approaches are recommended according to CENELEC standard?}

The use of formal approaches is \emph{Highly Recommended} for SIL3 and
SIL4 software according to CENELEC EN 50128:2011.

\subsection{Which constraints are required on the use of formal  approaches ?}


\subreq{The techniques applied to the software will be compliant regarding the SIL.}


\subreq{The tools used shall be developed in order to be certifiable according to EN 50128.}

\begin{comment}
No requirement on the way of doing this. \emph{E.g.} to have a certified 
(certifiable?) code generator, two generators and comparison of the result, one
 generation and one verification chain\dots
 \end{comment}
 
\subsection{Which are the benefits to use formal approaches ? }


\begin{comment}
From D.2.5 :

The purpose of the formalization is:
\begin{itemize}
\item to enhance the understanding of modelled subset;
\item to allow formal analysis of the modelled subset;
\item to be able to animate the model for testing an analyzing purpose;
\item to provide information on the completeness and soundness of the SUBSET-26;
\item to be used as a reference formal specification for the implementation of an OBU 
(by the OpenETCS project team and by industrial actors);
\item \dots
\end{itemize}
\end{comment}

\subsection{What means are involved behind a formal approach ?}

\subsubsection{Model edition}

\subsubsection{Mathematical analyses}
static checking, proof, model-checking, ...

\subsubsection{Simulation and code generation}

