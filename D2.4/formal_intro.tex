

\section{Short introduction on formal approaches to  design and validate critical systems}

\subsection{What is a formal approach?}

A \emph{formal} approach is a way to describe system or software
that builds upon (i) rigorous syntax and (ii) rigorous semantics.

The \emph{syntax} defines how the system or software is described. It
is usually made through a grammar and a set of additional
constraints. It can be textual or graphical.

The \emph{semantics} roughly describes how the system or software
evolves along time. It is defined using a mathematical model, i.e. use
of mathematical objects attached to each element of the syntax and
mathematical rules that define how those objects evolve. The
mathematical models used can be very different from a formal approach
to another one. For example B~Method uses the Generalized
Substitutions, SCADE relies on the Synchronous language Lustre, etc.

A \emph{semi-formal} approach is one where the syntax is precisely
defined but the semantics is not precisely defined, usually through
some English text. Typical semi-formal approaches are Matlab language
or SysML/UML formalisms.

A semi-formal approach can become formal if its semantics is
rigorously defined through a mathematical model.

\subsection{When formal approaches are recommended according to CENELEC standard?}

The use of formal approaches is \emph{Highly Recommended} for SIL3 and
SIL4 software according to CENELEC EN 50128:2011.

\subsection{Which constraints are required on the use of formal approaches?}

Each formal approach has some restriction on the kind of software or
system it can be applied on. Moreover, each formal approach is
specialized in the verification of some kind of property. Therefore a
formal approach should be chosen in accordance to verification
objectives.

Moreover, using a formal approach can impact the overall system
building process. For example software developed using the B~Method
follow a specific process and imposes a very specific architecture,
very different from designing C software. In the same way, use of
formal approach can impose specific resource needs at different phases
of project lifetime. For example, more work on the requirement
analysis and formalization phase.

Last but not least, as a formal approach brings its benefits only
inside a given boundary, the process should be designed to keep those
benefits outside those boundaries. For example, code compilation of a
verified source code should be done in such a way as to ensure that
the verified properties are kept in the compiled code.

\subsection{Which are the benefits to use formal approaches?}

Several benefits are expected from the use of formal approaches.

The first benefit is to enhance the understanding of the formalized
system or software. By using a non ambiguous notation, the designer is
forced to clarify his mind. Very often, several design issues or
defects are found at this step.

The second benefit is to enable the verification of some properties in
an exhaustive way. Therefore avoidance of certain kinds of bugs can be
guaranteed. Of course, such guarantee can only be obtained if the
formal method is used along some specific way and on a well delimited
part of the software and system (for example one cannot guarantee
properties on variables outside program boundary).

The third benefit is to allow Correct by Construction software or
system building. By verification properties along the construction
cycle of a system or software, one can ensure that some formalized
requirements are fulfilled in the final software. For example, one can
ensure that some variables stay in well defined boundaries.

\subsection{What means are involved behind a formal approach ?}

\subsubsection{Model edition}

\subsubsection{Mathematical analyses}
static checking, proof, model-checking, ...

\subsubsection{Simulation and code generation}

