\documentclass{template/openetcs_article}
% Use the option "nocc" if the document is not licensed under Creative Commons
%\documentclass[nocc]{template/openetcs_article}
\usepackage{lipsum,url}
\usepackage{supertabular}
\usepackage{multirow}
\usepackage{color, colortbl}
\definecolor{gray}{rgb}{0.8,0.8,0.8}
\usepackage[modulo]{lineno}
\graphicspath{{./template/}{.}{./images/}}
\begin{document}
\frontmatter
\project{openETCS}

%Please do not change anything above this line
%============================

%%%%%555 Macro Definitions %%%%%%%%%

%Starts a new line almost anywhere
\newcommand{\nl}{\mbox{}\\}

%to remove text from the final document while keeping it in the source
\newcommand{\nthng}[1]{}

%To mark old text parts 
\newcommand{\oldtext}[1]{\textbf{OLD:} {\em #1} \textbf{DLO}}
%uncomment the follwing line for a final version
%\renewcommand{\oldtext}[1]{}

%in-line (short) comment
\newcommand{\cmmnt}[1]{\fbox{#1}}
%uncomment the following line for a final version
%\renewcommand{\cmmnt}[1]{}

%Long comment
\newcommand{\bgcmmnt}[1]{\nl\framebox{\parbox{.95\textwidth}{#1}}\nl[2mm]}
%uncomment the follwing line for a final version
%\renewcommand{\bgcmmnt}[1]{}

%Uncertain items
\newcommand{\qq}[1]{?`#1?}
%uncomment the follwing line for a final version
%\renewcommand{\qq}[1]{}

%Things left (temporarily) open
\newcommand{\tbd}{\cmmnt{tbd}}

%Background color of boxes in process graphic
\definecolor{light-gray}{gray}{0.95}

%%%%%%%%% End of Macro Definitions %%%%%%%%%

% The document metadata is defined below

%assign a report number here
\reportnum{OETCS/WP2/D2.3a}

%define your workpackage here
\wp{Work-Package 2: ``Requirements''}

%set a title here
\title{Definition of the openETCS Development Process}

%set a subtitle here
\subtitle{A Standard-Compliant Process for the EVC Software Development}

%set the date of the report here
\date{June 2015\\
Revised September 2015\\
2\textsuperscript{nd} Revision October 2015}


%document approval
%define the name and affiliation of the people involved in the documents approbation here
\creatorname{Hardi Hungar}
\creatoraffil{German Aerospace Center}

\techassessorname{Jan Welte}
\techassessoraffil{TU Brunswick}

\qualityassessorname{Peter Mahlmann}
\qualityassessoraffil{DB netz}

\approvalname{Klaus-R\"udiger Hase}
\approvalaffil{DB Netz}


%define a list of authors and their affiliation here


\author{Hardi Hungar}

\affiliation{German Aerospace Center\\
  Lilienthalplatz 7\\
  38108 Brunswick, Germany}

% \author{Marielle Petit-Doche}
% \affiliation{Systerel}

% \author{Matthias G\"udemann}
% \affiliation{Systerel}
  


% define the coverart
\coverart[width=350pt]{openETCS_EUPL}

%define the type of report
\reporttype{Technical Report}


\begin{abstract}
%define an abstract here
%  \lipsum[12-13]
  This documents describes a standard-compliant process suitable for
  the development of the software of the EVC (European Vital Computer)
  in the openETCS approach. It instantiates an illustrative life cycle
  from the EN~50128:2011 for the specific goal of openETCS. The
  EN~50128:2011 is the standard relevant for software development of
  safety-critical rail systems. The instantiation is complemented by
  additions concerning system aspects, i.e., issues beyond software
  development. These additions are in line with the EN~50126-1:1999,
  the standard addressing the development of safety-critical rail
  systems. 

  The main concretions and modifications to the software development
  as described in the EN~50128:2011 are:
  \begin{itemize}
  \item The definition of a  system phase which generates the input
    necessary for the development of an EVC as a sub-system of the
    ETCS system from available background material.
  \item A sub-system (EVC) phase defining an interface of the HW to
    enable a SW development without considering the HW in detail.
  \item A high-level definition of the choice of implementation
    language, techniques and measures for the SW development and its
    verification.
  \item A validation concept for the SW without prior SW/HW
    integration based on simulation.
  \item A sketch of an iterative implementation of the process phases
    (in contrast to a sequential implementation where each phase is
    completed before the next begins).
  \end{itemize}

\end{abstract}

%=============================
\maketitle

%Modification history
%if you do not need a modification history table for your document simply comment out the eight lines below
%=============================


\section*{Modification History}
\tablefirsthead{
\hline 
\rowcolor{gray} 
Version / Date & Section & Modification / Description & Author \\\hline}
\begin{supertabular}{| m{3.2cm} | m{1.2cm} | m{4.6cm} | m{4cm} |}
 01 / June 23, 2015 & all & initial & Hungar 
\\\hline
02 / September 02, 2015 & all & incorporated comments by Jan Welte and
completed the coverage of D2.3 according to the analysis by
Hardi Hungar 

\textbf{Additionally:} Changed responsibilities for component test. &
Hungar
\\\hline  
03 / October 26, 2015 & 2, 3.7 & Corrected Fig.~1, added verification to Phase 7 & Hungar
\\\hline 
\end{supertabular}


\tableofcontents
\listoffiguresandtables
\newpage
%=============================

%Uncomment the next line if you need line numbers for tracebility when the document is in review
%\linenumbers
%=============================


% The actual document starts below this line
%=============================

%Start here


\section{Introduction}
\label{sec:introduction}

\subsection{Document Context} 
\label{sec:documemnt-context}

\subsubsection{openETCS Project and openETCS Activity}
\label{sec:project-activity}

This document is a contribution to the ITEA~2 project within the
context of the openETCS activity.

Here, it is distinguished between the current \emph{openETCS project}, funded as
part of the EUREKA cluster programme ITEA~2, and the \emph{openETCS activity}
as a whole, which encompasses the project.

The openETCS activity pursues the vision of a full CENELEC compliant
development of open-source software for the \emph{European Vital
  Computer} (EVC). It is intended to produce a software kernel which
can be used in commercial EVCs. Also, a development method with
suitable tool support shall be defined. The project is a major first
step in realizing the goals of the activity. Though the project cannot
realize all of the vision, it shall produce partial results which can
subsequently be taken up to achieve that goal.

\subsubsection{Scope of the openETCS Development}
\label{sec:scope-development}

The starting point is the ETCS specification, mainly Subset~026
\cite{subset-026:3.3.0}. From that, and a hardware interface
description, the SW requirements are to be derived in a (logically)
first development phase. The result shall be a generic SW covering the
full functionality described in \cite{subset-026:3.3.0}, but without
additional features which are usually required for a product. Such
additional features include functions addressing national specifics or
customer extensions. Since there are no plans yet to deploy the SW to
a dedicated HW, the SW cannot be validated on the target HW as
foreseen by the CENELEC standards. Instead, the development ends with
a SW validation via simulation.

\subsubsection{Role of the Document}
\label{sec:role-document}

This document provides a high-level view of a suitable process for the
openETCS activity. The project, which does only part of the
development, shall follow the description given here. The results of
the project shall be artifacts of this process or parts of such
artifacts, and they shall be labeled accordingly. What is not done
completely shall be delineated. All of this shall result in a set of
artifacts which forms a useful basis for subsequent activities. These
activities shall be able, by addressing the missing points, to
complete a documented software development suitable for integration in
a system which can be certified.

This document updates and extends the definition of the openETCS
process given in \cite{openETCS:D2.3}. That specification dates from
June 2013 and does not cover completely the current views. D2.3 is
extended in providing more detail on how to structure the development
compliant to the CENELEC standards, and on the process steps
themselves. 

Some development aspects described in D2.3 are not covered here. They
concern mainly: 
\begin{itemize}
\item A semi-formal model of the ETCS system. This is not done in the
  project and not necessary for developing the EVC software.
\item Formal sub-models of the ETCS system, in particular of the EVC
  software functions. Some such activities are pursued in the project,
  but this is considered to be merely a specific kind of implementing
  process steps and thus need not be included in an overall process
  definition. 
\item Applications of formal methods, in particular for safety-related
  parts. Again, this is not done on a large scale and merely one of
  several approaches pursued. Thus, it is not necessary to be detailed
  here.
\end{itemize}
%
These items may be covered in a supplementary document describing
ambitious activities of openETCS following the current project.

\subsubsection{Reference Material}
\label{sec:reference-material}

The main reference for the openETCS process is the standard for
developing software for railway control and protection systems, the
EN~50128:2011 \cite{EN50128:2011}. For the system context, the
EN~50126-1:1999 \cite{EN50126:1999} is taken into account, which
regulates the specification and demonstration of reliability,
availability, maintainability and safety of railway applications in
general (of which SW is a part). In particular, this standard is
relevant for the phases before the SW development. The standard
EN~50129:2010 \cite{EN50129:2010}, which concerns the safety case, is
not relevant for this description. It has to be considered for the
implementation of all safety-related activities listed here.


From within the project, the definition of the openETCS requirements
from D2.6 \citep{openETCS:D2.6} are relevant for an elaboration of
Sec.~\ref{sec:proc-impl}, which sketches the implementation of the
openETCS process.

\subsection{Scope of the Document}
\label{sec:scope-document}

\subsubsection{Items Covered}
\label{sec:items-covered}

This document defines the structure of the SW development process of
openETCS and its stages. It references and summarizes the main
requirements originating from the standards. It lists artifacts to be
produced and activities to be performed. It includes an instantiation
of the system phases necessary to derive the software
requirements. This instantiation is done in a form tailored to
openETCS where several specifications are already available (mainly
the UNISIG documents). 


\subsubsection{Items not Covered}
\label{sec:items-not-covered}


The document does not include full detail on the means (methods and
tools) to be employed. This is described (at least for the concerns of
the ITEA~2 project) elsewhere:
\begin{enumerate}
\item The \emph{primary openETCS toolchain} for software design given
  by \cite{openETCS:D7.1} 
\item the \emph{secondary openETCS toolchain} for verification and
  validation described in \cite{openETCS:D4.1.1.2}
\end{enumerate}
%
It also does not address organizational issues of the SW development,
i.e.\ how roles like Designer, Verifier etc.\ are to be filled, as
defined in the EN~50128.  These organizational issues will have to be
addressed during activities following the ITEA~2 project. This will be
necessary for integrating the resulting SW into an actual product and
having the result approved.

The description of activities and requirements on artifact content
given here does not provide full detail. In planning and performing
any of the process steps defined here, it is recommended to consider
the respectively relevant standard(s) to ensure full standard
compliance. References to the standards are provided.

The document does not specify in which way the results of the current
openETCS project will not cover the full functionality of the EVC, and
how it is planned to deal with this incompleteness. Each design
artifact produced shall define its scope, including a statement of
what is missing for a full development.

\subsection{General Description of the openETCS Development Process}
\label{sec:general-description}

The openETCS SW development process follows the Illustrative
Development Lifecycle~2 \cite[Fig.~4]{EN50128:2011} of the EN~50128
and the detailed description for ``Generic software development'' in
Sec.~7 of that standard. It deviates mainly in start and end phases
from that lifecycle. These phases are adapted to the specifics of the
openETCS goals, as not a full system will be developed (no hardware)
and a substantial body of background material concerning the EVC and
its environment is available. Additionally, as s a minor deviation,
the SW component design phase has been made part of the SW
architecture and design phase. This deviation reflects the form of the
development within the openETCS project, but it is of not much
significance and could easily be undone.

\subsubsection{Terminology}

\emph{System} refers to the ETCS system as described in Subset~026
\cite{subset-026:3.3.0}. This includes the onboard unit, the European
Vital Computer (EVC). The term \emph{sub-system} means this onboard
unit. This terminology has been introduced in the first description of
the openETCS process \cite{openETCS:D2.3}. A list of abbreviations and
terms is given in Sec.~\ref{sec:glossary}.

\subsubsection{Presentation of the Process}
\label{sec:presentation-process}

The presentation starts with an overview of the process. 
For each of the phases of the process, the following aspects are
described. These same aspects are defined in the standard.
\begin{description}
\item[Objectives] The goals of the phase
\item[Activities] Description of what has to be done
\item[Input] Lists artifacts and external source material
\item[Output] Process artifacts which are either produced or
  substantially altered by the activities
\item[Requirements] Short recapture of the
  requirements of the EN~50126 or EN~50128 for that phase, and how
  they are to be met by the deliverables.\footnote{Usually, the recapture does
    not provide full detail. Thus, one has to refer to the standards
    when planning the activities of a phase.}
\end{description}
%
The process is described as a top-down, waterfall procedure. The
implementation within the openETCS project deviates from that, as the
phases of the process are performed in an overlapping, iterative
style. How that is done is indicated in a separate section
(Sec.~\ref{sec:proc-impl}). The documentation of the development which
is to be produced shall not reflect the iterative implementation of
the process, but describe its results as if they had been achieved in
the top-down, waterfall way of the description in
Sec.~\ref{sec:devel-lifecycle} (as foreseen in
\cite[7.1.2.2]{EN50128:2011}). Then, a completed set of artifacts
would be in accordance with the main requirements of the CENELEC on
the documentation.\footnote{Additionally, it will be necessary to
  address organizational requirements adequately.}




\section{List of Artifacts and Phases}
\label{sec:list-artifacts}
This section lists all main artifacts which are to be produced to
document the development and provide the basis for a safety case of a
product using the openETCS project results. Further detail on the
content and role of these artifacts is given in the text describing
the phases where these artifacts are to be produced or updated.

The artifacts in the list are numbered in the format
``\textit{m}-\textit{n}'', where ``\textit{m}'' is the number of the
phase where the first version of the artifact is produced, and
``\textit{n}'' is a unique number (roughly the sequence number the
artifact would get in a linear implementation of the process). For
each artifact, the role mainly responsible for its creation is defined
in accordance with \cite[Tab.~C.1]{EN50128:2011}. The roles are taken
from \cite[Sec.~5.1 / Fig.~2]{EN50128:2011} and listed below. These
roles have also been assigned to the artifacts related to the system phases,
i.e.\ the scope of the \cite{EN50126:1999} which does not mention
roles. If two roles are given for one artifact, they are responsible
for different parts.
%
\begin{description}
\item[PM] Project Manager
\item[RQM] Requirements Manager
\item[DES] Designer
\item[IMP] Implementer
\item[INT] Integrator
\item[TST] Tester
\item[VER] Verifier
\item[VAL] Validator
\item[CM] Configuration Manager 
\end{description}
%
% This list contains short definitions of the document content. More
% detail is given in the pases where these documents are produced.
%
Note that implementing independence requirements according to
\cite[5.1.2.10]{EN50128:2011} is necessary for a SIL-4 development. As
this is not done in the project, subsequent activities with the goal
of producing code usable in a SIL-4 system will have to
care for that.

Generally, the Validator checks the results of the Verifier, and vice
versa. This is made explicit for the planning phase, where the
Validator has to check, besides the verification report, the
``Verification Plan'' written by the Verifier.
%
\begin{description}
\item[Phase 0: Planning]
\item[0-00 Project Plan (PM)] A management structure for the development,
  including a documentation of the assignment of roles and personnel to activities.
\item[0-01 Quality Assurance Plan (PM)] Definition of all activities to
  ensure the quality of the openETCS development (RAMS, functionality
  etc.). This document, which defines the lifecycle, is a part of the Quality
  Assurance Plan.
\item[0-02 Configuration Management Plan (CM)] A document which details how to handle
  versions (documents, software, tools) in the development.
\item[0-03 Verification Plan (VER)] Definition of all verification
  activities and how they are to be documented, and the methods and
  tools for verification.
\item[0-04 Validation Plan (VAL)] Definition of how to demonstrate that the end result (EVC
  software) serves its purpose within the context of the ETCS system.
\item[0-05 Planning Verification Report (VER, VAL)]
  Documentation of the verification of the planning documents.
\item[Phase 1: System Design] 
\item[1-06 Documentation of Coverage by Background Material (PM,RQM)] A
  listing of relevant available background material and a
  documentation of what is used for which purpose in openETCS. The RQM is
  concerned with the requirements, the PM handles all else. 
\item[1-07 Elaborated System Requirements (RQM)] Corrections and additions to
  background requirement definitions, including
  \begin{itemize}
  \item System and sub-system functionality
  \item System RAMS policy
  \end{itemize}
  Together with requirements from the background material, this shall
  define all system requirements which are necessary to derive
  the sub-system software requirements. 
\item[1-08 Risk Assessment (PM)] Documentation of the results of the
  analysis and evaluation of risks, including a hazard log for safety management.
\item[1-09 Safety Plan (PM)] Definition of activities to ensure safety of
  the openETCS development result (in the context of the ETCS system). 
\item[1-10 Sub-System Requirement Specification (RQM)] The
  requirements on the EVC.
\item[1-11 Sub-System Safety Specification (RQM)] The safety
  requirements on the EVC.
\item[1-12 System Design Verification Report (VER)] Documentation of the verification
  of all activities and results of the first openETCS process phase.
\item[Phase 2: Sub-System Architecture Design] 
\item[2-13 Sub-System Architecture Design (DES)] The SW/HW
  architecture of the EVC is described. The requirements and safety requirements are
  attributed to the components of the architecture.
\item[2-14 SW Acceptance Plan (VAL)] Definition of acceptance criteria and
  acceptance procedure for the EVC SW.  
\item[2-15 Sub-System Architecture and Design Verification Report (VER)]
  Report on the verification of the results of Phase~2.
\item[Phase 3: SW Specification] 
\item[3-16 SW Requirements Specification (REQ)]  A complete and consistent
  specification of the SW requirements, including the safety
  requirements in a specific section. 
\item[3-17 Overall SW Test Specification (TST)] Specification of tests
  checking the conformity of the sub-system software to its specification. 
\item[3-18 SW Specification Verification Report (VER)] Report on the
  verification of the results of Phase~3.
\item[Phase 4: SW Design] 
\item[4-19 SW Architecture and Design Specification (DES)]  This
  document combines SW architecture and a detailed SW design. It contains a
  definition of the coding principles and rules.
\item[4-20 SW Interface Specification (DES)] A detailed specification
  of the interfaces of the overall SW and of the SW components. This
  complements the ``SW Architecture and Design Specification''.
\item[4-21 SW Integration Test Specification (INT)] Specification of
  tests and test procedure to check that the SW components interact
  correctly.
\item[4-22 SW Component Test Specification (TST)] Specification of
  white-box tests on module level and black-box tests on component level.
\item[4-23 SW Design Verification Report (VER)] Report on the
  verification of the results of Phase~4.
\item[Phase 5: SW Component Implementation and Test] 
\item[5-24 SW Components (IMP)] The SW components, provided as SCADE models
  with code generated from them, and documented white-box tests.
\item[5-25 SW Component Test Report (TST)] Documentation of the component tests. 
\item[5-26 SW Component Verification Report (VER)] Report on the
  verification of the results of Phase~5.
\item[Phase 6: SW Integration] 
\item[6-27 SW Integration Test Report (INT)] Documentation of the
  integration test. 
\item[6-28 SW Integration Verification Report (VER)] Report on the
  verification of the results of Phase~6.
\item[Phase 7: SW Validation] 
\item[7-29 Overall SW Test Report (TST)] Report on the conformity test of
  the SW.
\item[7-30 SW Validation Report (VAL, VER)] Report on the validation of the
  SW, including the validation of the tools used in the process. 
\end{description}




\section{The openETCS Development Lifecycle}
\label{sec:devel-lifecycle}

% \begin{figure}[hbt]
%   \centering
%   \includegraphics[width=.9\textwidth]{Lifecycle-50128.png}
%   \caption{openETCS Development Lifecycle (to be replaced by graphic of the lifecycle tailored to openETCS)}
%   \label{fig:lifecycle}
% \end{figure}

% Fig.~\ref{fig:lifecycle} gives an overview of the openETCS development
% lifecycle. The phases are detailed below.

\begin{figure}[hbt]
  \centering
  \def\svgwidth{.9\textwidth}
  {\tiny
  \input{Prcss2_3a-03.pdf_tex}}
  \caption{openETCS Development Lifecycle}
  \label{fig:lifecycle2}
\end{figure}

Fig.~\ref{fig:lifecycle2} gives an overview of the openETCS
development lifecycle. It has been derived from Fig.~1 from
\cite{EN50128:2011}. The main differences to that standard consist in:
\begin{itemize}
\item A system phase has been added which covers the activities preceding
  the sub-system development (starting from the sub-system requirements)
\item The ``SW architecture and design phase'' and the ``SW component
  design phase'' have been combined into one phase. This is purely
  organizational. All information of the standard approach still have
  to be generated.
\item The ``SW component implementation phase'' and ``SW component
  testing phase'' have been combined into one. This is motivated by
  changing the responsibility for performing component tests to the
  Implementer and letting the Tester only check the results. This
  decision is in accordance with \cite[6.1.4.1 to
  6.1.4.5]{EN50128:2011} (see below). But nevertheless it might be
  reconsidered in the process implementation.
\item According to the standard, SW validation should happen
  \emph{after} SW/HW integration. Since it would contradict the goals
  of openETCS to fix the hardware, the SW cannot be deployed to the
  target HW and thus not be validated on it. To validate the SW within
  openETCS, the best option seems to be to do validation within
  simulated hardware.
\item Phases after SW validation are outside the current scope of
  openETCS, though a process for SW maintenance should be installed in
  the long run.
\end{itemize}

The phases are detailed below. For each phase, its objectives and the
activities to reach these objectives are given. Inputs and outputs are
listed, the latter usually with some more detail than in the table
above. Finally, all requirements of the relevant standard on the
respective phase are summarized together with a reference to the full
text. If the phase selects a particular alternative, a rationale is
given.

%%%%%% Phase 00 %%%%%%%%%
\addtocounter{subsection}{-1}
\subsection{Planning}
\label{sec:planningPhase}

\subsubsection{Objectives}
\label{sec:0-objectives}
Generate first versions of the planning documents.

\subsubsection{Activities}
\label{sec:0-activities}

Review the ideas and concepts of the openETCS approach in view of the
standard requirements (EN~50126 and EN~50128). Design a lifecycle for
the realization of the EVC software and define the activities (design,
verification, validation). Make preliminary choices for methods and
tools. The resulting documents are verified (VER), the verification
itself is checked (VAL).

\subsubsection{Input}
\label{sec:0-input}
The CENELEC standards, UNISIG specifications and the ITEA~2 project
application documents \cite{openETCS:FPP_4.0}.

\subsubsection{Output}
\label{sec:0-output}
First versions of the following documents.
\begin{enumerate}
\item \textbf{Project Plan (PM)} A management structure for the
  development, including a documentation of the assignment of roles
  and personnel to activities. For details, it refers to other
  planning documents. Deviating from the standard requirements, in the
  openETCS project there is no fixed assignment of personnel to
  roles. The project outcome is not intended to be taken directly to
  SIL 4, so this is not strictly necessary. It will have to be
  considered in ensuing activities.
\item \textbf{Quality Assurance Plan (PM)} Definition of all
  activities to ensure the quality of the openETCS development (RAMS,
  functionality etc.). This plan has supplements providing details on
  the lifecycle (this document), the tools and methods for design, modeling and
  implementation, as well as the necessary tool qualification.
\item \textbf{openETCS Process (PM)} (this document) This is a
  supplement to the ``Quality Assurance Plan'' and provides a consistent,
  high-level view on a standard-compliant development process,
  taylored to the goals and approach of the openETCS activity. 
\item \textbf{Definition of the tools and methods for software
    development (PM)}: This is a supplement to the ``Quality Assurance
  Plan''. It defines the openETCS ``Primary Tool Chain'': Methods,
  languages and tools for SW design and implementation, and their
  qualifications.
\item \textbf{Configuration Management Plan (CM)} A document which
  details how to handle versions (documents, software, tools) in the
  development.
\item \textbf{Verification Plan (VER)} Definition of all verification
  activities (checks to establish the adequacy, completeness and
  correctness of each phase) and how they are to be documented. It
  also addresses methods and tools.
\item \textbf{Validation Plan (VAL)} Definition of how to demonstrate
  that the end result (EVC software) serves its purpose within the
  context of the ETCS system. The plan includes a description the
  means to perform the validation and the techniques employed.
\item \textbf{Planning Verification Report (VER,VAL)}
  Documentation of the verification of the planning documents. This is
  done by the Verifier except for the ``Verification Report'', which is
  checked by the validator.
\end{enumerate}
%
All of these documents are to be updated in later phases of the
development to include additions and concretions and to reflect any
changes which are made in the plans. All changes shall be made in the
plans before the respective activities are performed. Concretions will
concern in particular, but not exclusively, tools and methods for
development, verification and validation.


\subsubsection{Requirements}
\label{sec:0-requirements}

The requirements to be met are defined mainly in \cite[Sec.~5,
7.1]{EN50128:2011}, with additional detail on SW quality assurance in
Sec.~6.

 \begin{description}
\item[Req 0.1, {\cite[Sec.~5.1]{EN50128:2011}}:] Organisation, roles and
  responsibilities are described in the ``Project Plan'' and ``Quality
  Assurance Plan''.
\item[Req 0.2, {\cite[Sec.~5.2]{EN50128:2011}}:] Personnel competence is
  to be addressed in development activities taking up the results of
  the openETCS project. 
\item[Req 0.3, {\cite[Sec.~5.3]{EN50128:2011}}:] The lifecycle is defined
  in this document.
\item[Req 0.4, {\cite[Sec.~7.1, 5.3.2.5]{EN50128:2011}}:] The phases are
  implemented in an iterative, overlapping fashion (see
  Sec.~\ref{sec:proc-impl}. All activities which are performed in the
  development are planned beforehand.
\item[Req 0.5, {\cite[Sec.~6]{EN50128:2011}}:] The quality assurance
  requirements shall be regarded in writing the plans for quality
  assurance, verification and validation.
 \end{description}

%%%%%% Phase 01 %%%%%%%%%
\subsection{System Design}
\label{sec:systemPhase}

\subsubsection{Objectives}
\label{sec:1-objectives}

The main goal of this phase is to generate everything that is
necessary to start the development of the sub-system, i.e.\ the EVC,
from the available input (Subset~026 and further material). This phase
shall cover the first five phases of the RAMS lifecycle of the ETCS
system as defined according to the EN~50126:
\begin{enumerate}
\item Concept
\item System definition and application conditions
\item Risk analysis
\item System requirements
\item Apportionment of system requirements
\end{enumerate}
The combination of five phases of the EN~50126 life cycle into only
one is motivated by the fact that not a full development needs to be
performed for openETCS. First, much is already available in background
documents. Further, only what is relevant to the EVC software has to
be provided for the subsequent development. Thus, it suffices to
complete the items which are inherited (at least partly) by the
requirements on the EVC SW, and those items which are related to the
integration of the SW into the ETCS system (e.g., interfaces and
assumption that are made about the rest of the system).


\subsubsection{Activities}
\label{sec:1-activities}

\paragraph{Identification of Coverage by Background Material}
\label{sec:1-a-0}

The available background material to cover requirements of the
EN~50126 is identified. ``Documentation of Coverage by Background
Material'' provides a detailed description of how this material covers
requirements. Sec.~\ref{sec:1-requirements} below lists what is
expected to be covered. For each instance where this expectations
turns out to be wrong, the respective requirement has to be met by
some dedicated activity. The outcome shall be referenced in the
``Documentation of Coverage by Background Material''.

\paragraph{System Requirement Elaboration}
\label{sec:1-a-1}

The available requirement specification of the ETCS system (mainly
given by Subset~026) relevant to the EVC is identified, analyzed and
extended by corrections and formalizations. This activity shall
produce the ``Elaborated System Requirements''. The analysis may use
semi-formal or formal modeling and analyses of the models.

\paragraph{Sub-System Requirement Specification}
\label{sec:1-a-2}

A definition of the sub-system boundaries and interfaces is derived,
and those of the system requirements that are to be allocated at least
partially to the sub-system are identified. From that, a sub-system
requirement specification is developed. This specification may be
text-based, or involve formalizations (e.g., models) as available. A
Sub-System Safety Specification shall be a separate document or a
separate section. The safety specification is based on risk analysis
activities. Output of this activity are ``Sub-System Requirement
Specification'' and ``Sub-System Safety Specification''. 

\paragraph{System Risk Analysis and Sub-System Risk Identification}
\label{sec:1-a-3}

Hazards and risks of the ETCS system are identified. This must cover
at least those which involve the EVC, and the way these are related to
hazards and risks in the full ETCS system. Existing sources for such
analyses may be used. The results shall be documented as part of the
Risk Analysis and Hazard Log.

\paragraph{Sub-System Safety, Acceptance and Quality Assurance Planning}
\label{sec:1-a-4}

This planning shall take into account that openETCS is concerned with
the software of the sub-system. It is therefore sufficient to
delineate the plans on the sub-system level. These delineations shall
be parts of the respective plans of openETCS, i.e., the Safety Plan,
Acceptance Plan and Quality Assurance Plan.

\paragraph{Verification}

The results of this phase shall be verified w.r.t.:
\begin{itemize}
\item selection of the right input information
\item adequacy of the methods and tools used in the derivation
\item correctness
\item adequacy and completeness for the respective purpose
\end{itemize}
 

\subsubsection{Input}
\label{sec:1-input}

The relevant input includes the UNISIG specification documents
\cite{subset-026:3.3.0, subset-034:3.0.0, subset-076:2.3,
  subset-088:2.3.0, subset-091:3.2.0}, which are based on the
Commission Decision 2012/88/EU on the technical specification for
interoperability relating to the control-command and signalling
subsystems of the trans-European rail system
\cite{unisig_CCS}. Additional sources may come from national
instantiations of ETCS track side equipment (e.g., Deutsche Bahn,
SNCF, ProRail) and operating rules.

\subsubsection{Output}
\label{sec:1-output}

\begin{enumerate}
\item \textbf{Documentation of Coverage by Background Material (RQM)}
  A listing of relevant available background material and a
  documentation of what is used for which purpose in openETCS. These
  purposes include:
  \begin{enumerate}
  \item  System scope, context, purpose and environment
  \item System RAMS targets, policy
  \item System hazards and risks
  \end{enumerate}
\item \textbf{Elaborated System Requirement Specification (RQM)}
  Corrections and additions to background requirement definitions,
  including
  \begin{enumerate}
  \item System and sub-system functionality
  \item System RAMS policy
  \end{enumerate}
\item \textbf{Sub-System Requirement Specification (RQM)}  The
  requirements on the EVC. This artifact defines the
  interface and the data flow over the interface, functionality of the
  EVC, and extra-functional requirements. It may refer to external
  documents for parts of this definition.  
\item \textbf{Risk Assessment} (PM)  Documentation of the results of the
  analysis and evaluation of risks, including a hazard log for safety
  management. The project manager has to organize the risk analysis
  and evaluation leading to the creation of the ``Risk Assessment''
  document, engaging all necessary parties. The ``Risk Assessment''
  shall include:
  \begin{enumerate}
  \item A documentation of the system risk analysis and evaluation performed
  \item An identification of risks relevant for the sub-system
  \item A hazard log for the sub-system
  \end{enumerate}
\item \textbf{Sub-System Safety Specification (RQM)} The safety
  requirements on the EVC.
\item Sub-System Safety Plan (part of the ``Safety Plan)''
\item Sub-System Acceptance Plan (part of the ``Acceptance Plan'')
\item Sub-System Quality Assurance Plan (part of the ``Quality Assurance Plan'')
\item Sub-System Hazard Log (part of the ``Sub-System Safety Specification'')
\item \textbf{System Design Verification Report (VER)} Documentation
  of the verification of all activities and results of the first
  openETCS process phase.
\end{enumerate}

\subsubsection{Requirements}
\label{sec:1-requirements}

The following text lists the requirements as defined in the EN~50126
for those 50126 phases included in this openETCS phase. For each
requirement it is said whether it is (expected to be) covered by
background material, or the designated outputs and the activities
described above. Requirements of the EN~50126 are referenced by their
identifying numbers. ``\textbf{Req~6.1.3.4}'' refers to the item
\textbf{6.1.3.4} in the EN~50126. This is ``Requirement~\textbf{4}''
of the phase ``Concept'', described in Sec.~\textbf{6.1} of the
EN~50126, whose requirements are listed in Sec.~\textbf{6.1.3}.


% In the EN~50126, requirements are numbered consecutively, restarting
% with ``1'' in each phase. In this document, they are grouped according
% to their 50126 phase and the requirement number is given. To resolve a
% reference of the form ``Req~\textit{n}'', one has to look up the phase
% in the EN~50126 first and then look for ``\textit{n}'' in the ``Requirement

The coverage by background material shall be detailed in
``Documentation of Coverage by Background Material'', if necessary
with adequate justifications. A general requirement is that full
bidirectional traceability between the requirements in the UNISIG
specifications (those which refer to the sub-system) and the
``Sub-System Requirements Specification'' and ``Sub-System Safety
Specification'' shall be realized.

\paragraph{Phase ``50126~6-1~Concept''}

As the ETCS system has already been implemented in numerous instances,
most items of the Concept Phase are expected to be covered by
background material. 
%
\begin{description}
\item[Req 1.1, {\cite[6.1.3.1]{EN50126:1999}}] System scope, context,
  purpose and environment: Background
\item[Req 1.2, {\cite[6.1.3.2]{EN50126:1999}}] RAMS implications of
  financial aspects and feasibility studies: Background
\item[Req 1.3, {\cite[6.1.3.3]{EN50126:1999}}] Hazard sources:
  Background
\item[Req 1.4, {\cite[6.1.3.4]{EN50126:1999}}] Information about RAMS
  performance, regulations and safety targets: Background
\item[Req 1.5, {\cite[6.1.3.5]{EN50126:1999}}] Definition of the
  management requirements: ``Quality Assurance Plan''
\end{description}


\paragraph{Phase "50126~6.2~System definition and application conditions"}


\begin{description}
\item[Req 1.6, {\cite[6.2.3.1]{EN50126:1999}}] Mission profile,
  boundary, application conditions and hazards of the system:
  Background
\item[Req 1.7, {\cite[6.2.3.2]{EN50126:1999}}] Preliminary RAM
  analysis and hazard identification: Background
\item[Req 1.8, {\cite[6.2.3.3]{EN50126:1999}}] RAMS policy for the
  system: Background
\item[Req 1.9, {\cite[6.2.3.4]{EN50126:1999}}] System Safety Plan: To
  be included in the ``Safety Plan'' (to the extent which is
  necessary)
\end{description}

\paragraph{Phase ``50126~6.3~Risk analysis''}

A risk analysis is performed for all items which concern the EVC SW
and how they relate to risks and hazards of the full system. A hazard
log with all relevant items enters the Sub-System Safety
Specification. The risk analysis may be based on background material.

\paragraph{Phase ``50126~6.4~System requirements''}

\begin{description}
\item[Req 1.10, {\cite[6.4.3.1]{EN50126:1999}}] Specification of RAMS
  requirements: Covered by ``Elaborated System Requirements''
\item[Req 1.11, {\cite[6.4.3.2]{EN50126:1999}}] Specification how to
  achieve compliance with the RAMS requirements: ``Safety Plan'',
  ``Validation Plan'' and ``SW Acceptance Plan''
\item[Req 1.12, {\cite[6.4.3.3]{EN50126:1999}}] Detailed RAM
  Programme: ``Verification Plan'', ``Quality Assurance Plan''
\item[Req 1.13, {\cite[6.4.3.4]{EN50126:1999}}] Updated Safety Plan:
  ``Safety Plan''
\end{description}

\paragraph{Phase ``50126~6.5~Apportionment of system requirements''}

\begin{description}
\item[Req 1.14 {\cite[6.5.3.1]{EN50126:1999}}] Allocate system
  requirements to sub-systems: The ``Sub-System Requirement
  Specification'' and the ``Sub-System Safety Specification'' address
  the parts relevant to openETCS, the rest should be covered by
  background material
\item[Req 1.15 {\cite[6.5.3.2]{EN50126:1999}}] Acceptance criteria for
  all sub-systems: the openETCS-relevant part is to be spelled out in
  the ``Acceptance Plan''
\item[Req 1.16 {\cite[6.5.3.3]{EN50126:1999}}] Updated Safety and
  Validation Plan: ``Safety Plan'' and ``Validation Plan''
\end{description}


%%%removed - relation to D2.3 shall be handled externally
% \subsubsection{Relation to Previous Descriptions}
% \label{sec:3-relations}

% This phase comprises to Steps 0 to 2 of, \cite{openETCS:D2.3}. The
% \emph{External System Requirement Identification} is the output of
% Step~0 (openETCS Inputs). 

% \tbd



%%%%%% Phase 02 %%%%%%%%%
\subsection{Sub-System Architecture Design}
\label{sec:sub-system-AD}

\subsubsection{Objectives}
\label{sec:2-objectives}
This phase shall complete the preparations for the main focus of
openETCS, the development of the EVC software. It shall define the
SW/HW architecture, the SW/HW interface and interaction, and 
allocate the system and system safety requirements to the
architecture components.

The last sub-phase of ``System Design'', the ``Apportionment of system
requirements'', allocated requirements to the sub-system. Here, these
are further refined to the SW/HW architecture of the sub-system. The
subsequent phase ``SW Specification'' completes the derivation of SW
requirements (of the EVC) from the system requirements (of ETCS). 

\subsubsection{Activities}
\label{sec:2-activities}
The hardware architecture of the system is described, and the SW/HW
interface is defined.  The hardware is described at least generically,
so that the it can be shown that the EVC functionality can be realized
on that architecture. The requirements and safety requirements which
concern the software are identified. Risk analysis and planning
documents are revised as necessary.

\subsubsection{Input}
\label{sec:2-input}

The main phase-specific inputs are:
\begin{enumerate}
\item ``Sub-System Requirement Specification''
\item ``Sub-System Safety Specification''
\item UNISIG specifications of the architecture of the EVC
\end{enumerate}

\subsubsection{Output}
\label{sec:2-output}

\begin{enumerate}
\item \textbf{Sub-System Architecture Design Specification (DES)}: The
  SW/HW architecture of the EVC is described. The requirements and
  safety requirements are attributed to the components of the
  architecture. The HW architecture is described on a high level. Its
  interface to the SW is given by a detailed HW API, describing in
  detail the cooperation and data flow between SW and HW.
\item \textbf{SW Acceptance Plan (VAL)} Definition of acceptance criteria and
  acceptance procedure for the EVC SW.  
\item \textbf{Sub-System Architecture Design Verification Report (VER)}: The
  verification shall check that the specification describes a
  realizable design, that the requirement allocation is complete
  and consistent with the architecture details, and that all aspects,
  in particular safety, have been adequately taken care of.
\end{enumerate}

\subsubsection{Requirements}
\label{sec:2-requirements}

The requirements to be covered come from the EN~50126, Phase~5,
``Apportionment of system requirements''. 
\begin{description}
\item[Req 2.1, {\cite[6.5.3.1]{EN50126:1999}}] Allocation of
  requirements (also safety) to sub-systems and specification of the
  sub-systems: The allocation is done in the ``Sub-System Architecture
  Design Specification''. The HW is described in its structure, and
  its interface to the SW is specified in detail. This enables a
  specification of the SW (including the safety aspect) in the
  subsequent phase. As the hardware is not in the focus of openETCS,
  the interface description is sufficient for the further openETCS
  development.
\item[Req 2.2, {\cite[6.5.3.2]{EN50126:1999}}] Specification of
  acceptance criteria for the architecture components. For the SW,
  this is done in the subsequent phase. The HW must provide the
  specified interface (API) to the SW. Besides that, detailed
  compliance criteria for the HW are to be provided by entities
  developing it. This is sufficient for the purposes of openETCS, as
  the rest of the system is outside the scope of openETCS.
\item[Req 2.3, {\cite[6.5.3.3]{EN50126:1999}}] Reviews and updates to
  the ``Safety Plan'' and ``Validation Plan''. These are done to the
  extent necessary (see ``Activities'' above).
  \end{description}

%%%%%% Phase 03 %%%%%%%%%
\subsection{SW Specification}
\label{sec:software-requirements}

As mentioned above in Sec.~\ref{sec:2-objectives}, this phase
completes what is to be done in Phase~5 of the EN~50126. In the
EN~50128, the SW requirement are defined in a separate phase. We
adopt this structuring, here. It goes well with the general idea of
the openETCS project, which focuses the SW development.  

\subsubsection{Objectives}
\label{sec:3-objectives}
A complete and consistent set of requirements, with specifically
designated safety requirements, for the SW shall be defined. This is
to be complemented by a definition of acceptance criteria for the SW. 


\subsubsection{Activities}
\label{sec:3-activities}

An analysis of the requirements which have been allocated completely
or partially to the SW in the ``Sub-System Architecture Design
Specification'' is done. The goal of the analysis is to extract the SW
requirements. In addition to the requirement documents produced in
previous process steps, the background material referenced in
``Documentation of Coverage by Background Material'' is a source for
requirements on the SW. 

Test cases are collected which cover functionality and performance of
each function, and in operational scenarios the interplay of the
functions.

\subsubsection{Input}
\label{sec:3-input}

\begin{enumerate}
\item ``Documentation of Coverage by Background Material''
\item Background material: UNISIG specifications
  \cite{subset-026:3.3.0, subset-034:3.0.0, subset-076:2.3, subset-088:2.3.0,
    subset-091:3.2.0} and further material as listed in
  the documentation of coverage by background material.
\item ``Elaborated System Requirement Specification''
\item ``Sub-System Requirement Specification''
\item ``Sub-System Safety Specification''
\item ``Sub-System Architecture Design Specification''
\end{enumerate}

\subsubsection{Output}
\label{sec:3-output}

\begin{enumerate}
\item \textbf{SW Requirements Specification (DES)} A complete and consistent
  specification of the SW requirements, including the safety
  requirements in a specific section. 
\item \textbf{Overall SW Test Specification (TST)} Specification of tests
  checking the conformity of the sub-system software to its specification. 
\item \textbf{SW Specification Verification Report (VER)} Report on the
  verification of the results of SW Requirements Phase.
\end{enumerate}

\subsubsection{Requirements}
\label{sec:3-requirements}


The requirements are derived from \cite[Sec.~7.2.4]{EN50128:2011}
and adapted to the EVC SW.
\begin{description}
\item[Req 3.1] The ``SW Requirements Specification'' shall 
  \begin{description}
  \item[a] address functionality, robustness, maintainability, safety,
    efficiency, portability, self checking, testing in operation, 
  \item[b] be complete, clear, precise, unequivocal, verifiable,
    testable, maintainable, feasible, back traceable, understandable
  \item[c] list all interfaces, all HW-related issues and modes of
    operation 
  \item[d] distinguish safety-related from other functions
  \item[e] use formalizations for complex logical and numerical
    functionalities and semi-formal modeling for structural representations  
  \end{description}
\item[Req 3.2] The ``Overall SW Test Specification'' shall
  \begin{description}
  \item[a] include functional black-box tests based on a boundary value
    analysis and apply input partitioning according to equivalence
    classes of inputs according to the specification
  \item[b] include performance tests for response times
  \item[c] be specified in more detail in the ``Verification Plan''
    with rationales for the adequacy of the chosen measures
  \end{description}
\item[Req 3.3] The ``SW Specification Verification Report'' shall be
  specified in the ``Verification Plan''.
\end{description}


%%%%%% Phase 04 %%%%%%%%%
\subsection{SW Design}
\label{sec:softw-arch-desi-phase}

\subsubsection{Objectives}
\label{sec:4-objectives}

A structure of the SW is defined, with components whose combination
will realize the SW requirements. The interplay of the components is
specified, in a form that it can be checked on the
implementation. Safety requirements are treated with adequate care so
that the contribution of each component to the safety function of the
SW is unambiguously clear and safety is verifiable. The description
must enable implementors to independently develop components and
provide them with references to all means to do so.

\subsubsection{Activities}
\label{sec:4-activities}

The SW requirements are analyzed to generate a suitable decomposition
into components realizing subfunctions. An execution paradigm for the
whole SW and the interplay of components (timing, execution order,
communication paradigm) is defined. The SW interfaces are specified
precisely. Languages, coding rules and tools for the implementation
are chosen. A test strategy for integration testing is chosen and the
integration and component tests are specified. The adequacy and
correctness of the results is verified.


\subsubsection{Input}
\label{sec:4-input}

The main phase-specific inputs are:
\begin{enumerate}
\item \textbf{SW Requirements Specification}
\item \textbf{Sub-System Architecture Design} provides the external
  interface of the SW via the HW API definition.
\end{enumerate}

\subsubsection{Output}
\label{sec:4-output}

\begin{enumerate}
\item \textbf{SW Architecture and Design Specification (DES)} This
  artifact covers SW architecture, overall SW design and detailed
  component design in the form of detailed specifications. All
  functionality is described in its realization by independent
  functions. The coding principles and rules for implementing the
  components are defined. In particular, this shall address the use of
  the SCADE Suite.  The artifact may be split over several documents.
\item \textbf{SW Interface Specification (DES)} This complements the
  ``SW Architecture and Design Specification'' by providing detailed
  interface specifications of the overall SW and of the SW components. 
\item \textbf{SW Integration Test Specification (INT)} Specification
  of the procedure for integrating the SW, testing that the integrated
  SW components interact correctly, and that the SW can run on the
  specified HW. The SW integration test may reuse tests from the
  component tests. The integration with HW is tested via a HW
  simulation.
\item \textbf{SW Component Test Specification (TST)} On the level of
  modules, the specification defines a general procedure for white-box
  testing which is to be instantiated with adequate tests for each
  module. The procedure includes a coverage criterion for SCADE models
  which is to be achieved by the tests. On the level of components,
  black-box tests checking the correct implementation of the
  functionality of the components are specified.
\item \textbf{SW Design Verification Report (VER, VAL)}
  Report on the verification of the results of this phase.
\end{enumerate}

\subsubsection{Requirements}
\label{sec:4-requirements}

\begin{description}
\item[Req 4.1 {\cite[7.3.4.2 to 7.4.3.17]{EN50128:2011}}] The ``SW
  Architecture and Design Specification'' defines the structure of the
  SW addressing feasibility, HW/SW interaction, components,
  scheduling, the paradigm for component communication and
  interaction, safety, fault handling, the use of prototype models,
  development strategy, techniques and measures. No pre-existing SW
  will be used. The architecture measures to be taken according to
  \cite[A.3]{EN50128:2011} are Defensive Programming, Diverse
  Programming, Fully Defined Interfaces, Modeling and Structured
  Methodology. \emph{Rationale: This is one of the approved
  combination (``1) a)'' with ``Modeling'') of techniques. To use
  ``Error Detecting Codes'' as in combination ``1) b)'' would not make
  sense, so ``1) a)'' is chosen. ``Modeling'' is a technique which is
  applied in the openETCS approach, so it is taken as the
  complementary technique for ``1) a)''.}
\item[Req 4.2 {\cite[7.3.4.18 to 7.4.3.20]{EN50128:2011}}:] The ``SW
  Interface Specification'' addresses input and output invariants
  (including bounds) and measures in case of invariant violation. The
  communication mechanism are detailed in the SW architecture and design.
\item[Req 4.3 {\cite[7.3.4.21 to 7.4.3.24, 7.4.4.1 to
    7.4.4.6]{EN50128:2011}}:] The overall SW and component design
  uses, from \cite[A.4]{EN50128:2011}, Modeling, a modular approach,
  design and coding standards and a strongly typed
  language. \emph{Rationale: This is one of the approved combination
    of techniques, with ``Modeling'' chosen instead of the alternative
    ``Formal Methods''.}  Components shall have fully defined
  interfaces with a general strategy of explicitly specified ways of
  data access. \emph{Rationale: This is in accordance with the
    requirements of \cite[A.20]{EN50128:2011}.}  The SCADE Suite
  Advanced Modeler is used for implementation. Its language is
  strongly-typed.  The style of modeling for the software and
  component design is to be defined in accordance with
  \cite[A.17]{EN50128:2011}. The component design shall relate SW
  requirements to those components which contribute to the
  requirement's implementation (with precise detailing of the safety
  aspect) so that both the component requirements are defined clearly
  and requirements can be traced, verified and tested. The way the
  components are implemented shall be sketched (main internal data
  representations and, where applicable, algorithms). Each component
  specification shall be detailed so that it can be independently
  implemented, i.e., not requiring knowledge of the implementation of other
  components. This entails a definition of the overall and specific
  cooperation principles between components and their interfaces. 
\item[Req 4.4 {\cite[7.3.4.25 to 7.4.3.28]{EN50128:2011}}:] A modeling
  guideline for SCADE models shall be defined which transfers accepted
  criteria for SIL-4 coding \cite[A.12]{EN50128:2011} to the level of
  SCADE models with their LUSTRE semantics. A style for documentation
  shall be defined.
\item[Req 4.5 {\cite[7.3.4.29 to 7.4.3.32]{EN50128:2011}}:] The ``SW
  Integration Test Specification'' shall employ Dynamic Analysis and
  Testing, Functional/Black-box Testing and Performance Testing
  \cite[A.5,A.6]{EN50128:2011}. \emph{Rationale: This combination is
    approved for SIL~4. Performance testing, which is highly
    recommended, is included as it seems necessary.} A precise
  description of form and content of the test specification shall be
  given in the ``Verification Plan''.
\item[Req 4.6 {\cite[7.3.4.33 to 7.3.4.39]{EN50128:2011}}:] These
  requirements concern the SW/HW integration. Since no
  hardware is being developed, the SW/HW integration will be replaced
  by simulating the SW in an adequate environment, e.g., with
  simulated HW. This shall be
  done in the SW Validation Phase and addressed in the ``Validation Plan''.
\item[Req 4.7 {\cite[7.4.4.7 to 7.4.4.10]{EN50128:2011}}:]
  For each component, the ``SW Component Test Specifications'' shall
  employ Dynamic Analysis and Testing, Test Coverage for Code,
  Functional/Black-box Testing and Performance Testing
  \cite[A.5]{EN50128:2011}.  A precise description of form and content
  of the test specifications shall be given in the ``Verification
  Plan''.
\item[Req 4.8 {\cite[7.3.4.40 to 7.3.4.43, Sec.~6.1.4]{EN50128:2011}}:]
  The verification of the architecture and design shall check that all
  SW requirements are allocated, all SW/HW interaction aspects are
  considered, that the architecture and design describe an adequate
  decomposition into subfunctions. It shall be verified that the
  aptness of the SCADE Suite in its use according to the coding
  principles and rules is sufficiently justified, and that the
  measures as a whole are in accordance with the EN~50128. It shall
  employ Static Analysis and Tracing to verify the suitability and
  completeness of the test specifications.
\end{description}

%%%%%% Phase 05 %%%%%%%%%
\subsection{SW Component Implementation and Test}
\label{sec:sw-component-implementation}

\subsubsection{Objectives}
\label{sec:5-objectives}

This phase shall provide verified implementations of the components
via modeling and code generation in the SCADE tool suite. 

There may be additional code components developed by other
means. The following text (Sec.~\ref{sec:5-activities} to
\ref{sec:5-requirements}) does not explicitly address these, but shall
apply analogously to them.  



\subsubsection{Activities}
\label{sec:5-activities}

The components are modeled in the SCADE Suite. Code is generated from
the models with the SCADE Suite code generator. The code is verified
as described in the component test specification. This includes the
generation and execution of white-box tests according to the procedure
defined in ``SW Component Test Specification'' to reach the required
coverage. Black-box test cases are to be derived from ``SW
Component Test Specification'' (addressing functionality and
performance) and executed. The component design, implementation and
test shall be independently verified. 

\subsubsection{Input}
\label{sec:5-input}

The main phase-specific inputs are:
\begin{enumerate}
\item ``SW Architecture and Design Specification''
\item ``SW Interface Specification''
\item ``SW Component Test Specification''
\end{enumerate}

\subsubsection{Output}
\label{sec:5-output}

\begin{enumerate}
\item \textbf{SW Components (IMP)} The software components in the form
  of SCADE models and C-code generated from them with the SCADE
  KCG. The modules are tested for basic functionality and
  compatibility with interface requirements 
\item \textbf{SW Component Test Report (TST)} Documentation of white-box and black-box
  tests according to the ``SW Component Test Specification'', as
  performed by the Tester. Test environment and test cases are
  to be provided so that the tests can be rerun. 
\item \textbf{SW Component Verification Report (VER)} Report on the
  verification of the results of this phase.
\end{enumerate}

\subsubsection{Requirements}
\label{sec:5-requirements}

\begin{description}
\item[Req 5.1 {\cite[7.5.4.1 to 7.5.4.4]{EN50128:2011}}:] The SCADE
  models shall have balanced size and complexity and be readable,
  understandable and testable. Details on how to achieve this are to
  be given in specific modeling guidelines. By testing, the
  Implementer shall assert that a component produced is mature enough
  for SW integration. \emph{Rationale: It is more efficient to let the
  programmer perform basic tests than to involve a third party. This
  also offers the possibility to improve code quality by testing
  partially integrated SW, prior to rigorous verification.}
  the code to a maturity level 
\item[Req 5.2 {\cite[6.1.4.5, 7.5.4.7]{EN50128:2011}}:] The test report
  shall document all essential information about the test object,
  tester, results, coverage and evaluation. This report addresses the
  verification activities performed by the Tester. Details shall be defined
  in the verification plan. 
% The Tester shall review the white-box
%   tests performed by the Implementer, and shall perform the black-box
%   component tests. \emph{Rationale: It is usually more efficient to
%     let the Implementer perform the white-box tests. Note that tests
%     performed by the Implementer are acceptable if they are fully
%     documented (specification and report) and have been performed with
%     suitable equipment, see \cite[6.1.4.1 to 6.1.4.5]{EN50128:2011}}.
\item[Req 5.3 {\cite[7.5.4.8 to 7.5.4.10]{EN50128:2011}}:] The
  verification of the results (documented in the ``SW Component
  Verification Report'') shall address the usage of the coding rules in
  the actual models, check structure, shape and documentation of the
  models and control that all required tests are documented.
\end{description}


%%%%%% Phase 06 %%%%%%%%%
\subsection{SW Integration}
\label{sec:sw-integration-test}


\subsubsection{Objectives}
\label{sec:6-objectives}

The integration shall result in a functional, integrated SW with
correctly interacting components.  

\subsubsection{Activities}
\label{sec:6-activities}

The SW components are integrated according to the integration
strategy. Concrete test cases for integration tests are constructed as
specified in the ``SW Integration Test Specification'' and applied to
check the correct interplay of the components.

\subsubsection{Input}
\label{sec:6-input}

\begin{enumerate}
\item ``SW  Integration Test Specification''
\item ``SW Components'': the objects to be integrated.
\item ``SW Component Test Specification'', ``SW Component Test
  Report'': For use in constructing test cases for the integration tests
  and evaluating the results (reuse  of unit tests).
\end{enumerate}

\subsubsection{Output}
\label{sec:6-output}

\begin{enumerate}
\item \textbf{SW Integration Test Report (INT)} Documentation of the
  integration test.
\item \textbf{SW Integration Verification Report (VER)} Report on the
  verification of the SW integration and test.
\end{enumerate}

\subsubsection{Requirements}
\label{sec:6-requirements}

\begin{description}
\item[Req 6.1 {\cite[7.6.4.1, 7.4.6.6]{EN50128:2011}}:] Necessary
  changes identified in the integration process have to be analyzed
  for their impact and trigger also re-verification. The tests shall
  be fully documented and be repeatable, and it shall be justified that an appropriate set of
  techniques and measures have been applied.\footnote{Inconsistently, the EN~50128:2011
    requires in 7.3.4.32 the ``SW Integration Test Specification'' to
    adhere to the requirements of Table~A.5, and in 7.6.4.6 to check
    that the techniques and measures from Table~A.6 have been
    correctly used. The procedure here follows Table~A.5 which subsumes Table~A.6.} 
\item[Req 6.2 {\cite[7.6.4.7 to 7.4.6.10]{EN50128:2011}}:] These address
  the SW/HW integration which is not done in openETCS. Substitute
  activities via HW simulation are done in the subsequent phase.
\item[Req 6.3 {\cite[7.6.4.11 to 7.4.6.13]{EN50128:2011}}:] The
  verification shall check that all test according to the ``SW
  Integration Test Specification'' have been documented in the test
  report, and that the report meets the general requirements on a test
  report and the specific ones (\cite[7.6.4.1,
  7.4.6.6]{EN50128:2011}).
\end{description}


%%%%%% Phase 07 %%%%%%%%%
\subsection{SW Validation}
\label{sec:sw-validation}


\subsubsection{Objectives}
\label{sec:7-objectives}

Validation of the functionality, safety and performance of the SW and
its fitness for integration on a suitable HW.

\subsubsection{Activities}
\label{sec:7-activities}

The SW is tested against its requirements according to the
``Overall SW Test Specification'' and additional validation scenarios
provided by the Validator. HW compatibility is checked via simulation, for which an
environment is set up. The validation results are documented.

\subsubsection{Input}
\label{sec:7-input}

\begin{enumerate}
\item ``Overall SW Test Specification''
\item ``SW Components'' (integrated)
\item ``Validation Plan''
\item ``SW Acceptance Plan''
\end{enumerate}

\subsubsection{Output}
\label{sec:7-output}

\begin{enumerate}
\item \textbf{Overall SW Test Report (TST)}: Report on the conformity
  test of the SW.
\item \textbf{SW Validation Report (VAL, VER)}: Report on the validation of
  the SW, including the validation of the tools used in the
  process. Verified by the Verifier.
\end{enumerate}

\subsubsection{Requirements}
\label{sec:7-requirements}

\begin{description}
\item[Req 7.1 {\cite[7.7.4.1 to 7.7.4.4]{EN50128:2011}}:] The
  ``Overall SW Test Report'' shall document
  \begin{itemize}
  \item the tests performed according to the ``Overall SW Test
    Specification'',
  \item additional tests defined by the Validator, which may in
    particular address operational aspects or potentially occurring
    stress situations, and
  \item how the HW compatibility has been checked via simulation.
  \end{itemize}
%
\item[Req 7.2 {\cite[7.7.4.6 to 7.7.4.11]{EN50128:2011}}:] The ``SW
  Validation Report'' shall state that the development has been done
  in accordance with the ``Validation Plan'', and that all
  verification activities have followed the ``Verification Plan'' and
  meet the requirements of the EN~50126-1:1999, resp., the
  EN~50128:2011. It shall also contain a substantiated statement that
  the overall combination of techniques and measures is adequate and
  in accordance with the standards, and that all tools are qualified
  to their purpose. Any discrepancies and deviations are to be
  documented and why they are acceptable. It shall contain a
  conclusion of the fitness of the SW for its purpose under a
  specified set of assumptions on the HW.
\item[Req 7.3 {\cite[6.3.4.13 to 6.3.4.14]{EN50128:2011}}:] The ``SW
  Validation Report'' shall contain a report of the verification, by
  the Verifier, of the validation plan and the reported
  validation. The verification shall address general and specific
  requirements and in particular internal consistency. Verification of
  the validation plan shall precede the further validation activities.
\end{description}

\section{Process Implementation}
\label{sec:proc-impl}

The process will not be implemented in a top-down, sequential style,
where, e.g., specifications are completed and verified before the
software is designed. Instead, the phases may overlap in time. This
means, even software components may be designed before the SW
specification is finalized. However, the final documentation of the
development shall present a consistent, top-down view of the
result. To ensure that this goal is achieved, specific considerations
have to be given to the iterative aspect in the ``Project Plan'',
``QA-Plan'' and in particular the ``Configuration Management Plan'' .


Ideally, a plan should be made which sets the goals to be achieved
in the openETCS project. This consists in defining  
\begin{itemize}
\item the functionality to be realized, and
\item the completeness and maturity levels of the artifacts. 
\end{itemize}
The maturity could, for instance, be defined in classes like
``Draft'', ``Revised'', ``Final''. Completeness refers to
functionality and other, artifact specific issues. 

The plan should define a timeline which states which grade shall be
achieved at which stage, and add corresponding activities.

Besides this general process flow, also the means, i.e., languages and
tools, and their impact on the process should be addressed in a
detailed description of the process implementation. In particular, the
ramifications of the use of the SCADE tool suite deserve specific attention.

\section{Glossary}
\label{sec:glossary}
\begin{description}
\item[API] Application Programming Interface
\item[EVC] European Vital Computer
\item[FME(C)A] Failure Mode Effect (and Criticity) Analysis
\item[FIS] Functional Interface Specification
\item[HW] Hardware
\item[I/O] Input/Output
\item[OBU] On-Board Unit
\item[openETCS activities] the openETCS project and ensuing activities 
\item[openETCS project] the current project (ITEA~2)
\item[openETCS] refers to the project and initiative, without a
  particular focus on either
\item[PHA] Preliminary Hazard Analysis
\item[QA] Quality Analysis
\item[RBC] Radio Block Center
\item[RTM] RunTime Model
\item[Semi-formal language] Language with a formal syntax (and a
  semantics which need not be strictly formal)
  necessarily 
\item[SIL] Safety Integrity Level
\item[SRS] System Requirement Specification
\item[SSHA] Sub-System Hazard Analysis
\item[SSRS] Sub-System Requirement Specification
\item[strictly-formal language] Language with a formal syntax and a
  formal, mathematical semantics
\item[Sub System] This term denotes the EVC, whose software is the main focus of openETCS
\item[SW] Software
\item[System] This term denotes the ETCS system as described in Subset~026 \cite{subset-026:3.3.0}
\item[Test Case] Precise description of input/output to/from a test
  object, including a success criterion. Ready for execution but not
  necessarily already in machine readable form.
\item[Test Specification] A description of a set of test
  cases. Not all parameters need to be specified precisely.
\item[THR] Tolerable Hazard Rate
\item[V\&V] Verification \& Validation
\item[Vital] attribute of safety related items, e.g.\ artifacts
\end{description}

%% Bibliography
% \nocite{*}
\bibliographystyle{unsrt}
\bibliography{Lbrr2_3a}



%Examples are below


%\lipsum[11]

%\nocite{*}

%\bibliographystyle{unsrt}
%\bibliography{Lbrr}



% \begin{thebibliography}{9}

% \bibitem{lamport94}
%   Leslie Lamport,
%   \emph{\LaTeX: A Document Preparation System}.
%   Addison Wesley, Massachusetts,
%   2nd Edition,
%   1994.

% \end{thebibliography}

%===================================================
%Do NOT change anything below this line

\end{document}
